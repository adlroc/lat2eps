
\documentclass[a4paper]{article}

\usepackage[utf8]{inputenc}
\usepackage[top=3cm,left=2.5cm,right=2.5cm,bottom=3cm]{geometry}
\usepackage{listings}
\usepackage{hyperref}

\title{\vspace{-5ex}lat2eps v2.0}
\author{}
\date{\vspace{-5ex}}

\begin{document}

\maketitle


\section{Introduction}

lat2eps is a small command-line utility for Linux, Unix and macOS that generates graphics in the Encapsulated PostScript (EPS) format, depicting the contents of square/rectangular lattices.
\bigbreak

lat2eps reads the lattice data from the standard input and outputs the EPS data to the standard output. The input file is a simple Gnuplot-compatible text file, where data lines contain at least 3 integer fields separated by spaces, where the first two are the horizontal and vertical coordinates of the lattice site, and the third is the site value. Lines beginning with the \# character are considered comments and ignored, unless they contain embedded internal commands, described below.
\bigbreak

There are 256 different color indexes, ranging from 0 to 255, which can be used to draw lattice sites of different values. Site values are mapped to color indexes according to a simple rule: $index = value \bmod 256$. Negative values are first converted to unsigned values (e.g., -1 is converted to 255, -2 to 254, and so on). Note: a palette of only 16 different colors is initially defined, repeated over the 256 possible indexes. Thus, if more than 16 colors are needed, they must be redefined, as described below.
\bigbreak


\section{Installation}

lat2eps is open source software supplied in source code form. It can be built with the \textit{make} command, which will generate the \texttt{lat2eps} executable, which can be manually copied to a directory within the user's path, like, e.g., \texttt{/usr/local/bin}. The \textit{gcc} or \textit{clang} compiler must be installed in the system for the build process to succeed.
\bigbreak


\section{Usage}

The basic command usage is:
\bigbreak

\texttt{lat2eps  xoff yoff width height border scale  <input.dat  >output.eps}
\bigbreak

Where \textit{xoff}, \textit{yoff}, \textit{width} and \textit{height} define offsets and dimensions (in sites) of a region within the lattice that will be used to generate the EPS output.
\bigbreak

\textit{border} is the width of a black border generated around the lattice graphic, or 0 for generating a borderless graphic.
\bigbreak

\textit{scale} is a positive integer value defining the scale used in the conversion of the lattice data to EPS. E.g., when the scale is set to 3, each lattice site will generate a 3x3 pixel square in the EPS output.
\bigbreak


\section{Embedded commands}
\bigbreak

Embedded commands can be placed within a lattice data file passed as input to lat2eps, inside comment lines, which begin with the \# character (the commands must immediately follow the \# character). Each embedded command can contain multiple parameters. The commands and their parameters can be separated by spaces or commas. The supported embedded commands are:
\bigbreak\bigbreak

\texttt{TXT <X>, <Y>, <AX>, <AY>, <ANGLE>, <SIZE>, <COLOR>, <TEXT>}
\bigbreak

The TXT embedded command is used to generate text entries over the lattice graphic. It can be used to generate text lines to appear in the graphic as they are, or tags that can be later replaced by LaTeX text using PSFrag. The parameters are the following:

\begin{itemize}
  \item \textit{X} is the horizontal coordinate where the text will be positioned. 0 is the leftmost coordinate, while the maximum horizontal coordinate is defined by the lattice width.
  \item \textit{Y} is the vertical coordinate where the text will be positioned. 0 is the topmost coordinate, while the maximum vertical coordinate is defined by the lattice height.
  \item \textit{AX} defines the horizontal alignment. 0 for left-aligning the text relative to the X coordinate, 0.5 for centering it on the X coordinate, 1 for right-aligning it, etc.
  \item \textit{AY} defines the vertical alignment. 0 for placing the top of the text on the Y coordinate, 0.5 for centering it on the Y coordinate, 1 for placing the bottom of the text on the Y coordinate, etc.
  \item \textit{ANGLE} defines the angle to rotate the text, in degrees (0 for horizontal).
  \item \textit{SIZE} defines the font size.
  \item \textit{COLOR} defines the color index used to draw the text.
  \item \textit{TEXT} is the text to be generated. It can contain any characters including spaces and commas; parentheses characters, however, must be escaped with backslashes.
\end{itemize} 
\bigbreak\bigbreak

\texttt{PAL <COLOR\_0\_PALETTE>, <COLOR\_1\_PALETTE>, <COLOR\_2\_PALETTE>, ...}
\bigbreak

The PAL embedded command can be used to redefine the color palette. It can receive a variable number of parameters. The first parameter will redefine the color index 0, the second will redefine color index 1, and so on. Each parameter is a hexadecimal number in the RRGGBB format (i.e., the first byte defines the red component for the color index, from 00 to FF, the second byte defines the green component, and the third byte defines the blue component).
\bigbreak\bigbreak

\texttt{COL <COLOR\_INDEX>, <COLOR\_PALETTE>}
\bigbreak

The COL embedded command can be used to redefine a single color index. It receives 2 arguments. The first is the color index, from 0 to 255, and the second is a hexadecimal color definition in the same format described for the PAL command.
\bigbreak\bigbreak

Sample lattice data file with embedded commands:
\bigbreak

\begin{lstlisting}[frame=single]
# Embedded commands here:
#TXT 128, 250, 0.5, 1, 0, 20, 1, X
#TXT 6, 128, 0.5, 0, 90, 20, 6, Y
#TXT 128, 128, 0.5, 1, 45, 20, 5, lat2eps test
#PAL 0, FFFFFF, 334455, 667788, 4433FF
#COL 2, 00ffff
#COL 5, 123456

# Lattice data:
0 0 2
1 0 1
2 0 1
3 0 2
4 0 2
5 0 3
\end{lstlisting}

\bigbreak

Embedded commands can also be supplied through the command line, as optional parameters to the lat2eps call, after the required parameters. In this case, they should not be preceded by the \# character, and the embedded commands and their parameters must be separated by commas only (or the spaces escaped), so that each embedded command and its parameters are recognized by the shell as a single parameter to the lat2eps command. For instance, to generate text and redefine color number 6:
\bigbreak

\texttt{lat2eps 0 0 320 320 1 1 TXT,25,25,0,0,0,20,6,Test COL,6,2245FF <lat.dat >lat.eps}
\bigbreak


\section{Default configuration}

Default configuration parameters can be supplied through a file named \texttt{.lat2epsrc}, placed within the user's home directory. This file can contain embedded commands, exactly in the same way as used in lattice data files, which can be used, for instance, to define a custom color palette to be used by default for the lattice graphics generated by the user.
\bigbreak

The color palette optionally defined in the defaults file redefines the initial lat2eps color definition. Afterwards, the lattice data input file is processed, which may contain embedded commands to further redefine the colors. After the lattice data is processed, embedded commands supplied through the lat2eps command line are processed, which may again redefine the colors.
\bigbreak


\section{License}

lat2eps v2.0 is open source software copyrighted by André R. de la Rocha and licensed under the Apache License Version 2.0. You may obtain a copy of the license at: \url{http://www.apache.org/licenses/LICENSE-2.0}

\end{document}

